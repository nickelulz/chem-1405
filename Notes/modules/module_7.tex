\section{Intermolecular Forces}

\subsection{Physical Properties}

\subsubsection{Disruptive Kinetic Energy vs. Cohesive Potential Energy}

\begin{defn}
Compressibility is a measure of the volume change resulting from a pressure change, and thermal expansion is a measure of the volume change resulting from a tempereature change.
\end{defn}

\noindent
The characteristics of physical properties (such as density, volume/shape, compressibility, or thermal expansion) are caused by the balance between two key energies within the molecules of each given substance: disruptive kinetic energy and cohesive potential energy. \\

\noindent
Simply put, disruptive kinetic energy can be described as the kinetic energy from the speed at which the particles within the substance travel, and cohesive potential energy can be described as the exact opposite, the bonding energy that arises from the lack of movement/potential for movement within the molecules of the substance. \\

\begin{itemize}
\item For solids, cohesive potential energy dominates over disruptive kinetic energy. As such, they have definite volume/shape, high density, low compressibility, and small thermal expansion.
\item For liquids, cohesive potential energy and disruptive kinetic energy are roughly equal. Thus, they have definite volume with indefinite shape, lower density than solids but higher density than gases,  and greater compressibility/thermal expansion than solids (yet still generally small).
\item For gases, disruptive kinetic energy dominates over cohesive potential energy. Thus, gases have indefinite volume/shape, low density, and high compressibility and thermal expansion.
\end{itemize}

\begin{table}[H]
\centering
\begin{tabular}{|l|l|l|l|l|}
\hline
State & Volume & Shape & Compressibility & Thermal Expansion \\
\hline
Solid & Definite & Definite & Very Small & Very Small ($0.01\%/^\circ C$) \\
Liquid & Definite & Indefinite & Small (but greater than solid) & Small ($0.10\%/^\circ C$) \\
Gas & Indefinite & Indefinite & Large & Moderate ($0.30\%/^\circ C$) \\
\hline
\end{tabular}
\end{table}

\subsubsection{Exothermic vs. Endothermic Changes}

\begin{defn}
Physical changes (changes between states of matter) can be described as \textit{endothermic} or \textit{exothermic}. Endothermic changes require the input of energy and feel cold to the touch. Exothermic changes release energy and feel warm to the touch.
\end{defn}

\subsubsection{Vapor Pressure}

\begin{defn}
Vapor pressure is the pressure exerted by vapor over a liquid in a sealed container at equilibrium, and it depends on the nature of the substance and its current temperature.
\end{defn}

\note{Volatile substances evaporate rapidly and have high vapor pressure.}

\subsubsection{Boiling Points}

\begin{defn}
The boiling point of substance is the temperature of a liquid at which the vapur pressure of the liquid becomes euqal to the external/atmospheric pressure exerted on the liquid.
\end{defn}

\noindent
The boiling point of a substance fluctuates with the atmospheric pressure, and it can be increased or decreased given changes to external pressure.

\subsection{Intermolecular/Intramolecular Forces (in Liquids)}

\begin{defn}
Intermolecular forces are attractions between molecules, and Intramolecular forces are the forces that hold an individual molecule together.
\end{defn}

\noindent
\note{Intermolecular forces play a role in determining the physical forces of a substance, such as meling point, boiling point, or shape.} \\

\noindent
\note{All intermolecular forces are weaker than bonding forces.}

\noindent
The strength order of intermolecular force types is H-Bonding (for polar molecules) $>$ Dipole-Dipole (for polar molecules) $>$ London Dispersion (for non-polar molecules).

\subsubsection{Dipole-Dipole Interactions}

\begin{defn}
Dipole-Dipole Interactions are attractions between polar molecules. As polar molecules are electrically uneven, they form diples that have positive and negative ends. These molecules naturally orient themselves where opposingly-charged ends are connected.
\end{defn}

\noindent
\textbf{Hydrogen Bonds} are an especially strong form of dipole-dipole interaction. For a Hydrogen bond to occur, Hydrogen must have been covalently boned to Fluorine, Oxygen, or nitrogen. From there, the Hydrogen in the bond can then further bond to another Fluorine, Oxygen, Nitrogen, or another Hydrogen Bond. (\note{The most common example of a compound with an H-Bond is Water.}) \\

\noindent
\note{The vapor pressure of liquids with significant hydrogen bonds are much lower than those of similar liquids with little to no H-Bonds and the boiling points of H-Bonded liquids are much higher than that of other substances, and as a result, H-Bonds make it far more difficult for their molecules to escape the liquid state.}

\subsubsection{London Dispersion Forces}

\begin{defn}
London Dispersion Forces are weak, temporary dipole-dipole interactions that occur because of momentary uneven electron distributions in non-polar molecules.
\end{defn}

\noindent
These are the most common intermolecular forces, and they are the only attractive forces in non-polar moles (and they account for 85\% of the intermolecular force in the polar HCl molecule. \\

\noindent
\note{London Forces only play a minor role if hydrogen bonding is present. Additionally, despite London forces being fleeting, they explain the extreme increase in elemental boiling points as you move down the periodic table.} \\

\noindent
The strength of London forces depends on how easy it is to distort the polarity of the molecule. Thus, the larger the diameter of the molecule, the easier it is to distort the polarity.

\begin{defn}
Polarizability is the ability to distort the electron cloud of an atom. it increases down a group of atoms or ions due to the increase in atomic radius and decreases from left to right across a period because of the increaseing effective nuclear charge.
\end{defn}

\noindent
\note{Cations are less polarizabile than their parent atom because they are smaller, whereas anions are more polarizable because they are larger.}

\section{Solutions}

\textbf{References Chapter 9 of the textbook.}

\subsection{Characteristics of Solutions}

\begin{defn}
In a solution, the solute is the substance dissolved in another substance, and the solvent is the substance that the solute is dissolved into. For example, in saltwater, salt would be the solute and water would be the solvent.
\end{defn}

\noindent
\note{The main factors that affect the solubility of solids in liquids are the nature of solvent, particle size, agitation/stirring, and temperature.} \\

\noindent
\note{The solute is always the least abundant component of the solution and the solvent is always the most abundant component of the solution.} \\

\noindent
There are nine types of two-component (solute-solvent) solutions:

\begin{itemize}
\item Gas dissolved in another gas (ex. dry air)
\item Liquid dissolved in a gas (ex. wet air)
\item Solid dissolved in a gas (ex. mothballs in the air)
\item Gas dissolved in a liquid (ex. soda)
\item Liquid dissolved in another liquid (ex. vinegar or cleaning solution)
\item Solid dissolved in a liquid (ex. saltwater)
\item Gas dissolved in a solid (ex. hydrogen in platinum)
\item Liquid dissolved in a solid (ex. amalgam dental fillings)
\item Solid dissolved in a solid (ex. sterling silver)
\end{itemize}

\noindent
\note{Many solids must be dissolved into a solvent (solution) to undergo appreciable chemical reactions.} \\

\noindent
\note{In a solution, solute poarticles are uniformly dispersed throughout the solvent.}

\begin{defn}
Solubility describes the maximum amount of solute that will dissolve in a specified amount of solvent, expressed as grams per solute per 100 grams of solvent.
\end{defn}

\subsection{Solution Terms}

\begin{table}[H]
\centering
\begin{tabular}{|c|c|}
\hline
Term & Range \\
\hline
Very Soluble & $S > 10g/100g$ \\
Soluble & $1g/100g < S \le 10g/100g$ \\
Slightly Soluble & $0.1g/100g < S \le 1g/100g$ \\
Insoluble & $S < 0.1g/100g$ \\
\hline
\end{tabular}
\caption{Solubility Descriptive Terms}
\end{table}

\noindent
For most solids dissolved in a liquid, an increase of termperature results in increased solubility. However, opposingly, as temperature increases, the solubility of a gas decreases.

\subsubsection{The Nature of Solvent}

\begin{defn}
The general rule for predicting solubility, known as the \textbf{Nature of Solvent}, is "like dissolves like," but this rule does not always work given the influence of the magnitude of the solute-solute interaction (or ion-ion forces) and solute-solvent interaction (or ion-dipole forces).
\end{defn}

\noindent
\note{Polar compounds tend to be more soluble in polar solvents than non-polar solvents.} \\

\begin{table}[H]
\centering
\begin{tabular}{|l|l|p{0.4\textwidth}|}
\hline
Ion(s) & Rule & Exceptions \\
\hline
\ce{Li+}, \ce{Na+}, \ce{K+}, \ce{NH4+} & Soluble & None \\
\ce{NO3-}, \ce{CH3CO2-} & Soluble & None \\
\ce{Cl-}, \ce{Br-}, \ce{I-} & Soluble (except with) & \ce{Ag+}, \ce{Hg2^{2+}}, \ce{Pb^{2+}} \\
\ce{SO4^{2-}} & Soluble (except with) & \ce{Ag+}, \ce{Hg2^{2+}}, \ce{Pb^{2+}}, \ce{Ca^{2+}}, \ce{Sr^{2+}}, \ce{Ba^{2+}} \\
\ce{S^{2-}}, \ce{O^{2-}}, \ce{OH-} & Insoluble (except with) & \ce{Li+}, \ce{Na+}, \ce{K+}, \ce{NH4+}, \ce{Ca^{2+}}, \ce{Sr^{2+}}, \ce{Ba^{2+}} \\
\ce{CO3^{2-}}, \ce{PO4^{3-}}, \ce{CrO4^{2-}} & Insoluble (except with) & \ce{Li+}, \ce{Na+}, \ce{K+}, \ce{NH4+} \\
\hline
\end{tabular}
\end{table}

\noindent
\note{If the solute-solute interactions, solute-solvent interactions, and the solvent-solvent interactions exhibit the same types of intermolecular forces at similar magnitudes, then the substance formed will be a homogeneous mixture/solution.}

\subsubsection{Aqueous and Non-Aqueous Solutions}

\begin{defn}
A non-aqueous solution is a solution in which a substance other than water is the solvent (such as alcohol-based solutions) and an aqueous solution is when water is the solvent.
\end{defn}

\subsection{Pressure Effects}

\noindent
The solubility of a gas in a liquid is a function of the pressure of the gas, where the higher the pressure, the greater the solubility of the gas. \\

\noindent
In order words, for gases, solubility and pressure are directly related. Where $S$ is solubility and $P$ is pressure, $S \propto P$

\subsection{Saturation}

At a specific temperature, there is a limit to the amount of solute that will dissolve in a given amount of solvent. When this condition occurs, the solution is said to be saturated.

\begin{defn}
A \textbf{saturated solution} is a solution that contains the maxium amount of solute that can be dissolved (under the conditions at which the solution currently exists). Similarly, an \textbf{unsaturated solution} contains less than this maximum, and a \textbf{supersaturated solution} contains less.
\end{defn}

\noindent
The terms dilute and concentrated are qualitiative expressions of the amount of solute present in a solution, where a dilute solution contains a relatively small amount of solute and a concentrated solution contains a relatively large amount of solute. \\

\subsubsection{Saturation Types}

\begin{itemize}
\item Mass percent is the mass of solute divided by the total mass of the solution multiplied by 100.
	\begin{itemize}
	\item Parts per million (ppm) can be expressed as 1 milligram per kilogram of solution, and it's equivalent to the mass of the solute divided by the total mass of the solution multiplied by $10^6$.
	\item Similarly, parts per billions (ppb) is $10^9$.
	\end{itemize}
\item Mass-volume percent is the mass of the solute divided by the total volume of solution multiplied by 100.
\item Volume percent is the volume of solute divided by hte total volume of the solution multiplied by 100.
\item The molarity (M) of a solution is the number of moles of solute per liter of the solution.
	\begin{itemize}
	\item Molarity can be calculated from grams of solute by converting moles of solute.
	\end{itemize}
\end{itemize}

\subsubsection{Dilution}

The moles of solute before dilution always equals the moles of solute after dilution.

\begin{equation}
M_{1} V_{1} = M_{2} V_{2}
\end{equation}

where the volume of solvent or water is equal to $V_{2} - V_{1}$.

\section{Acids and Bases}

\subsection{Properties of Acids and Bases}

\subsubsection{Acid Properties}

\begin{itemize}
\item Sour Taste
\item Changes the color or Litmus paper from Blue to Red
\item Reacts with metals like zinc and magnesium to produce hydrogen gas, carbonate to produce carbon dioxide, and release hydrogen ions in aqueous solutions.
\end{itemize}

\subsubsection{Basic Properties}

\begin{itemize}
\item Bitter or caustic taste (such as chocolate)
\item Changes the color of Litmus paper from Red to Blue
\item Dissolves fats (like drain cleaner)
\item Have a slippery, soapy feeling (but soap is a fatty acid salt, not a base).
\item Has the ability to react with acids
\end{itemize}

\subsection{Arrhenius Acid-Base Theory}

\note{Arrhenius was a Swedish scientists that advanced a theory of acids in bases in 1884.}

\begin{defn}
An arrhenius acid is a hydrogen-containing substance that undergoes ionization to produce hydrogen ions in an aqueous solution.
\end{defn}

\begin{example}
\ce{HA -> H+_{(aq)} + A-_{aq}}
\end{example}

\begin{defn}
An arrhenius base is a hydroxide-containing substance that dissociates to produce hydroxide ions in an aqueous solution.
\end{defn}

\begin{example}
\ce{MOH -> M+_{(aq)} + OH-_{(aq)}}
\end{example}

\subsubsection{Ionization and Dissociation}

\begin{defn}
Ionization is the production of ions from a molecular compounds that has been dissolved in solvent whereas dissociation is the production of ions from an ionic compound that has been dissolved in solvent.
\end{defn}

\subsection{Bronsted and Lowry Acid-Base Theory}

\begin{defn}
The Bronsted and Lowry Acid-Base Theory essentially defines two things: A B-L acid is a proton (\ce{H+}) donor, and A B-L base is a proton (\ce{H+}) acceptor.
\end{defn}

\begin{itemize}
\item You cannot have a B-L acid without a B-L base. The proton donor (acid) needs the proton acceptor (base).
\item All B-L acids are also Arrhenius acids, but not all B-L bases are Arrhenius bases.
\item The B-L acidic species in aqueous solution is not a Hydrogen ion (\ce{H+}), but rather, a Hydronium (\ce{H3O+}) ion.
\item B-L reactions can occur in the gaseous phase with no water required depending on the B-L acid/base involved.
\end{itemize}

\subsubsection{Conjugate Acids and Bases}

\begin{defn}
When an acid donates a proton, the species that remains is called the conjugate base, and when a base accepts a proton, the species it becomes is called the conjugate acid.
\end{defn}

\noindent
\note{In a Bonsted-Lowry equilibrium equation, the reactants and products can be grouped into two conjugate acid-base pairs.}

\begin{example}
For example, in \ce{HCl_{(g)} + H2O_{(l)} -> Cl-_{(aq)} + H3O+_{(aq)}}, \ce{HCl} is an acid that becomes \ce{Cl-}, a conjugate base, and \ce{H2O} is a base that becomes \ce{H3O+}, a conjugate acid.
\end{example}

\begin{itemize}
\item In a B-L acid-base pair, the acid has one more acidic H atom and one fewer negative charge than the base, and likewise, the base always has one fewer acidic H atom and one more negative charge than the acid.
\end{itemize}

\begin{defn}
An amphiprotic substance is a substance that can be either a B-L acid or a B-L base, such as \ce{H2O}.
\end{defn}

\subsubsection{Acidic Hydrogen}

\begin{defn}
Acidic hydrogen is the hydrogen atom in an acid molecule that is transferred as a proton in an acid-base reaction. It is always written at the beginning of the formula, separate from the non-acidic hydrogens (such as in \ce{HC2H3O2}), and in an oxyacid, it is bonded to oxygen.
\end{defn}

\noindent
Certain acids can have more than one type of acidic hydrogen. \textit{Monoprotic acids} such as \ce{HNO3} or \ce{HCl} only transfer one acidic hydrogen, but \textit{Polyprotic acids} such as \textit{Diprotic acid} (which transfer two hydrogens, such as \ce{H2SO4}) or \textit{Triprotic acid} (which transfer three hydrogens, such as \ce{H3PO4}). \\

\noindent
Some ionic compounds (salts) dissociate completely in aqueous solutions, and they are known as strong electrolytes. \note{At least one component of the salt will be from a strong acid or base.}

\begin{defn}
A strong acid is an acid that, in an aqueous solution, transfers 100\% (or close to 100\%) of its acidic protons to water, whereas a weak acid is an acid that, in an aqueous solution, only transfers a small percentage of its acidic protons to the water.
\end{defn}

\noindent
\note{Both the ionized and unionized forms of a weak acid (or any weak electrolyte) are present in an aqueous solution.}

\begin{table}[H]
\centering
\begin{tabular}{|l|l|}
\hline
Name & Molecular Formula \\
\hline
Nitric Acid & \ce{HNO3} \\
Sulfuric Acid & \ce{H2SO4} \\
Perchloric Acid & \ce{HClO4} \\
Chloric Acid & \ce{HClO3} \\
Hydrochloric Acid & \ce{HCl} \\
Hydrobromic Acid & \ce{HBr} \\
Hydroiodic Acid & \ce{HI} \\
\hline
\end{tabular}
\caption{Commonly Encountered Strong Acids}
\end{table}

\begin{table}[H]
\centering
\begin{tabular}{|l|l|l|}
\hline
Name & Molecular Formula & Group Number \\
\hline
Lithium Hydroxide & \ce{LiOH} & 1 \\
Sodium Hydroxide & \ce{NaOH} & 1 \\
Potassium Hydroxide & \ce{KOH} & 1 \\
Rubidium Hydroxide & \ce{RbOH} & 1 \\
Cesium Hydroxide & \ce{CsOH} & 1 \\
Calcium Hydroxide & \ce{Ca(OH)2} & 2 \\
Strontium Hydroxide & \ce{Sr(OH)2} & 2 \\
Barium Hydroxide & \ce{Ba(OH)2} & 2 \\
\hline
\end{tabular}
\caption{Commonly Encountered Strong Bases}
\end{table}

\subsection{Acid Reactions}

Acids react with metals, bases, carbonates, and bicarbonates. \\

\begin{itemize}
\item Acids react with metals that lie above hydrogen in the activity series of elements to produce hydrogen and an ionic compound (salt).
\item The reaction of an acid with a base is known as a \textit{neutralization} reaction. When it is a hydroxide base in an aqueous solution, the products are a salt and water.
\item Most acids react with carbonates to produce an ionic compound and carbonic acid (as a double replacement reaction). The carbonic acid then decomposes to water and carbon dioxide.
\end{itemize}

\noindent
\note{Even through water is a molecular (covalently-bonded) substance, a very small percentage of water molecules interact with one another to form ions. This process is known as the \textit{self-ionization} of water.} \\

\subsection{Power/Potential of Hydrogen}

\begin{defn}
The power/potential of hydrogen in an acid or base, represented as pH, is the negative logarithm of the hydronium ion concentration in the compound and acts as a quantitative measure of the acidity or basicity of aqueous (or likewise) solutions.

\begin{equation}
\mathrm{pH} = -\log[\ce{H3O+}]
\end{equation}
\end{defn}

\noindent
\note{Brackets in the equation represent the molar concentration of the compound.} \\

\begin{figure}[H]
	\centering
	\includegraphics[width=\textwidth]{ph_scale}
	\caption{The pH Scale}
\end{figure}

\begin{example}
\textit{What is the pH of a solution with an [\ce{H3O}+] (or molar concentration of \ce{H3O+}) of $1.0 \times 10^{-11}$ M?}

\begin{align}
\mathrm{pH} &= -\log\left[\ce{H3O+}\right] \\
\mathrm{pH} &= -\log\left[1.0 \times 10^{-11}\right] = -(-11) \\
\mathrm{pH} &= 11.00
\end{align}
\end{example}

\subsection{Buffers}

\begin{defn}
A buffer is a solution that resists major changes in pH when small amounts of acid or base are added to it, and buffer solutions are made of weak acid which can react and remove added bases and their conjugate bases which can react with and remove acid.
\end{defn}

\begin{example}
For example, \ce{HCN} and \ce{KCN} or \ce{HC2H3O2} and \ce{KC2H3O2} together can act as buffers in pairs.
\end{example}

\begin{itemize}
\item Buffers do not hold the pH constant.
\item pH change is less with a buffered solution than an unbuffered solution.
\item Buffer solutions are not all pH 7.
\end{itemize}

\begin{example}
Acid with a buffer:
\begin{equation}
\ce{H3O+ + C2H3O2- -> HC2H3O2 + H2O}
\end{equation}

\noindent
Base with a buffer:
\begin{equation}
\ce{OH- + HC2H3O2 -> C2H3O2- + H2O}
\end{equation}

\noindent
In summary, \ce{C2H3O2-} and \ce{HC2H3O2} exchange forms given the presence of \ce{H3O+} or \ce{OH-}.
\end{example}

\subsection{Acid-Base Titrations}

\begin{defn}
A titration is the process of measuring the volume of one reagent required to react with a measured mass or volume of another reagent. An indicator, such as Phenolphthalein, Bromythol Blue, or Litmus Paper, are compounds that exhibit different colors depending on the pH of their surroundings. 
\end{defn}

\begin{defn}
Acid-base titrations are a procedure where a measured volume of acid or base of known concentration is exactly reacted with a measured volume of a base or acid of unknown concentration.
\end{defn}

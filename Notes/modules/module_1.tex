\section{Lecture 1: Introduction}

\textbf{References chapter 1, sections 1.2 and 1.3 of the textbook.}

\vspace{2em}

\begin{minipage}{\textwidth}
\subsection{The Scientific Method}
\begin{defn}
The Scientific Method is the set of general procedures for acquiring scientific knowledge.

\begin{enumerate}
	\item Identify the problem
	\item Collect past data/perform background research
	\item Analyze and organize the data into general summaries of previous observations
	\item Suggest probable explanations/hypotheses for the generalizations
	\item Perform further experiments to prove or disprove these explanations/theorems
\end{enumerate}
\end{defn}
\end{minipage}

\begin{minipage}{\textwidth}
\subsection{Qualitative vs. Quantitative Data}
\begin{defn}
Qualitative vs. Quantitative Data
\begin{itemize}
	\item \textbf{Quantitative Data} - Numerical data yielded from \textit{measurements}.
	\item \textbf{Qualitative Data} - Any data yielded from other observation, \textit{including rough estimates}.
\end{itemize}
\end{defn}

\begin{itemize}
	\item The patient's fever has reached $105.3\mathrm{^\circ F}$ - \textit{Quantitative}
	\item The packet of Candy contains \textit{about} 100 gummy bears - \textit{Qualitative}
\end{itemize}
\end{minipage}

\begin{minipage}{\textwidth}

\vspace{2em}

\subsection{Scientific Facts, Laws, Hypotheses, and Theories}
\begin{itemize}
	\item \textbf{Scientific Fact} - Reproducible data/observations obtained from an experiment.
	\item \textbf{Scientific Law} - Summarizes scientific facts without interpretation. Accepted as fact but can be changed/contested.
	\item \textbf{Scientific Hypothesis} - Explains or contests scientific laws. Tested by experiments.
	\item \textbf{Scientific Theory} - Scientific hypotheses that are tested and validated as correct/true over a long period of time. \textit{Scientific Theories can never be proven true.}
\end{itemize}

\begin{example}
\begin{itemize}
	\textbf{Facts, Laws, and Theories}
	
	\item The burning candle generated both heat and light - \textit{Scientific Fact}
	\item As a candle burns, its wax gradually disappears - \textit{Scientific Fact}
	\item All burning candles generate heat and light - \textit{Scientific Law}
	\item Burning candles generate heat as the result of the decomposition of melted wax. - \textit{Scientific Hypothesis}
\end{itemize}
\end{example}
\end{minipage}

\section{Lecture 2: Measurement}

\subsection{Exact vs. Inexact and Accuracy vs. Precision}
\begin{defn}
Exact vs. Inexact and Accuracy vs. Precision
\begin{itemize}
	\item \textbf{Exact numbers} - Numbers with no uncertainty and are known exactly either in definition, counting, or as simple fractions (cannot be an irrational number). 
	\item \textbf{Inexact numbers} - Numbers with some degree of uncertainty.
	\item \textbf{Accuracy} - How close the measured values are to the intended/goal values
	\item \textbf{Precision} - How close the measured values are to each other
\end{itemize}
\end{defn}

\begin{itemize}
\item Measurements require units.
\item It is impossible to make exact measurements as all measurements have some degree of uncertainty due to limits of the measuring instrument (systematic error) and random error.
\end{itemize}

\noindent
When measuring, one has to keep the certain digits and uncertain digits in mind. The magnitude of uncertainty is indicated with a plus-minus notation alongside the acceptable range.

\subsection{Significant Figures and Rounding-Off Rules}
Significant figures are all the certain digits of measurement plus one digit of uncertainty. When determining significant figures for unknown measurements, the following rules are used:

\begin{itemize}
	\item Nonzero digits are always significant.
	\item Leading zeroes are never significant.
	\item Confined zeroes are always significant.
	\item Trailing zeroes are only significant if there is a decimal point or the zeroes carry overbars (i.e. are repeating).
\end{itemize}

\noindent
\note{Remember that the last significant figure is always uncertain.}

\begin{example}
Significant Figures
\begin{itemize}
	\item $14.232 \rightarrow 5 \text{ sig. figs, } \pm 0.001$
	\item $0.0045 \rightarrow 2 \text{ sig. figs, } \pm 0.0001$
	\item $2.075 \rightarrow 4 \text{ sig. figs, } \pm 0.001$
	\item $4300.00 \rightarrow 6 \text{ sig. figs, } \pm 0.01$
	\item $36,003 \rightarrow 5 \text{ sig.figs, } \pm 1$
	\item $6310 \rightarrow 3 \text{ sig. figs, } \pm 10$
\end{itemize}
\end{example}

\subsubsection{Rounding Rules}

\begin{enumerate}
	\item If the first digit to be dropped is less than five, then all proceeding digits are just dropped. (ex. $98.623 \rightarrow 98.6$)
	\item If the first digit to be dropped is greater than five or is a five followed by all zeroes, the previous digit is increased by one. (ex. $25.0566 \rightarrow 25.06$)
	\item If the first digit to be dropped is a five not followed by any other digit or a five followed by only zeroes, then the previous digit is increased by one only if it is odd. If it is even, it will not be incremented. (ex. $63.3500 \rightarrow 63.4$, but $62.65 \rightarrow 62.6$)
\end{enumerate}

\subsection{Significant Figures and Mathematical Operations}

\begin{itemize}
	\item For multiplication and division, always use the \textit{fewest significant figures.}
	\item For addition and subtraction, always use the number with the most uncertainty (or the number with the uncertainty digit in the highest place).
\end{itemize}

\begin{example}
Significant Figures with Operations
\begin{itemize}
	\item $6.038 \times 2.57 \approx 15.5$
	\item $677 + 39.2 + 6.23 \approx 722$
\end{itemize}
\end{example}

\subsection{Scientific Notation}
\begin{defn}
Scientific notation is a numerical system in which an ordinary decimal number is expressed as a product of a number between 1 and 10 and 10 raised to a power. (as in: $A \times 10^{N}$) The value $N$ refers to how many places the decimal of a number is moved, where $+N$ is right and $-N$ is left.
\end{defn}

\begin{example}
Scientific Notation
\begin{itemize}
\item $905000 = 9.05 \times 10^{5}$
\item $0.0006030 = 6.030 \times 10^{-4}$ \\ \note{The trailing zero is kept as it is significant.}
\item $1.894 \times 10^{-7} =0.0000001894$
\item $1.327 \times 10^{5} = 132700$
\end{itemize}
\end{example}

\section{Lecture 3: Unit System and Dimensional Analysis}

\subsection{The Metric System}
\begin{table}[H]
\centering
\begin{tabular}{|c|c|c|c|}
\hline
Quantity & Metric System & Customary System & Tool for Measure \\
\hline
Mass & Grams (g) & Pounds (lbs) & Scale \\
\hline
Length & Meters (m) & Inches (in.)/Various & Ruler \\
\hline
Volume & Liters (L) & Fluid Ounces (fl. oz.)/Various & Graduated Cylinder \\
\hline
Temperature & Celsius ($^\circ$C) & Fahrenheit ($^\circ$F) & Thermometer \\
\hline
\end{tabular}
\end{table}

\noindent
\note{The metric system is based on base 10, meaning that it follows prefixes that get multiplied to each number by a series of ten.}

\begin{table}[H]
\centering
\begin{tabular}{|c|c|c|c|}
\hline
Prefix & Symbol & Value \\
\hline
Tera & T & $10^{12}$ \\
\hline
Giga & G & $10^{9}$ \\
\hline
Mega & M & $10^{6}$ \\
\hline
Kilo & k & $10^{3}$ \\
\hline
Hecto & h & $10^{2}$ \\
\hline
Deca & da & $10^{1}$ \\
\hline
Deci & d & $10^{-1}$ \\
\hline
Centi & c & $10^{-2}$ \\
\hline
Milli & m & $10^{-3}$ \\
\hline
Micro & $\mu$ & $10^{-6}$ \\
\hline
Nano & n & $10^{-9}$ \\
\hline
Pico & p & $10^{-12}$ \\
\hline
\end{tabular}
\end{table}

\noindent
\note{This can be remembered with: \textit{The Great Mega King Henry Doesn't [usually] Drink Chocolate Milk Mixed [with] Nana's Peanuts}. \textbf{It is incredibly important that these values and their prefixes are remembered!}}

\subsection{SI Units}

SI is the international system of units, and denotes the standard units for each measurement.

\begin{itemize}
	\item SI Length is in Meters (m) ("Metres" internationally but "Meters" domestically).
	\item SI Mass is in Kilograms (kg). \note{Mass is the total number of matter, weight is mass multiplied by the force of gravity.}
	\item SI Volume is in Liters (L). ("Litres" internationally but "Liters" domestically), where 1 L = 1000 $\mathrm{cm^{3}}$.
	\item SI time is in seconds (s).
	\item SI temperature is in Kelvin (K).
	\item SI Current is in Amperes (amp).
	\item SI Amount/Count is in Moles (mol).
\end{itemize}

\noindent
\note{When measuring the volume of a solid, use $V = LWH$. For liquids in a graduated cylinder, read from the bottom of the meniscus.}

\vspace{2em}

\begin{minipage}{\textwidth}
\subsection{Metric to Metric Conversion}

\begin{enumerate}
	\item Determine what the end goal unit is.
	\item Determine what the given unit is.
	\item Multiply the given unit by the ratio fraction of the two units, where the initial unit's value is the denominator and the ending unit's value is the numerator.
	\item Round the final product to the original number of significant figures. \\
\end{enumerate}
\end{minipage}

\noindent
For English-to-Metric conversions, you will be given the conversion rates between English and Metric conversions. That is all that you need to know for that.

\subsection{Common Equation Reference}
\begin{itemize}
\item Density: $\rho = \ds \frac{\text{Mass}}{\text{Volume}}$
\item Percent Error: $\ds \frac{\text{Measured Value} - \text{Accepted Value}}{\text{Accepted Value}} \times 100$ (\note{Percent error can be negative, as the sign indicates the relation between the measured and the intended value.})
\item Rate or Velocity: $\ds v = \frac{\text{Distance}}{\text{Time}}$
\item Concentration: $\ds M = \frac{\text{Moles}}{\text{Volume}}$
\end{itemize}

\begin{minipage}{\textwidth}

\subsection{Densities of Common Substances at 25 degrees Celsius}

\begin{multicols}{2}
\begin{table}[H]
\centering
\begin{tabular}{|c|c|}
\hline
Substance & Density (g/mL) \\
\hline
Ethanol & 0.789 \\
\hline
Acetone & 0.791 \\
\hline
Rubbing Alcohol & 0.786 \\
\hline
Motor Oil & 0.899 \\
\hline
Olive Oil & 0.920 \\
\hline
Water & 1.00 \\
\hline
Seawater & 1.03 \\
\hline
Honey & 1.45 \\
\hline
Bromine & 3.12 \\
\hline
Mercury & 13.6 \\
\hline
\end{tabular}
\end{table}

\begin{table}[H]
\centering
\begin{tabular}{|c|c|}
\hline
Substance & Density (g/cm$^{3}$) \\
\hline
Balsa Wood & 0.160 \\
\hline
Cork & 0.210 \\
\hline
Magnesium & 1.74 \\
\hline
Aluminum & 2.70 \\
\hline
Iron & 7.86 \\
\hline
Copper & 8.92 \\
\hline
Silver & 10.5 \\
\hline
Lead & 11.34 \\
\hline
Gold & 19.32 \\
\hline
Platinum & 21.45 \\
\hline
\end{tabular}
\end{table}
\end{multicols}

\end{minipage}

\vspace{1em}

\noindent
\note{Cubic cenimeters equal milliliters.}

\vspace{2em}

\subsection{Temperature Conversions}
\begin{multicols}{2}
\begin{itemize}
\item $\ds T_{K} = T_{C} + 273.15^\circ$
\item $\ds T_{C} = \frac{5}{9}(T_{F} - 32)$
\item $\ds T_{F} = \frac{9}{5}T_{C} + 32$
\end{itemize}
\begin{itemize}
\item $273^\circ K = 0^\circ C$
\item $373^\circ K = 100^\circ C$
\item $32^\circ F = 0^\circ C$
\item $212^\circ F = 100^\circ C$
\end{itemize}
\end{multicols}

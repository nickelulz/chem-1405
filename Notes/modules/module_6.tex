\section{Lecture 1: Counting Atoms in Moles}

\subsection{The Law of Definite Proportions/Constant Composition}

\begin{defn}
The law of definite proportions (also known as the Law of Constant Composition) states that in a pure compound, the elements described by its formula are present in the same definite proportion by mass, and thus, it will always retain the same physical and chemical properties for all objects of this substance type.
\end{defn}

\noindent
For example, water always contains only two elements, Hydrogen and Oxygen, and it will always be about 11.2\% Hydrogen and 88.8\% Oxygen by mass. In the event that a compound differs from these qualities whatsoever, it is no longer water.

\subsection{Percent Composition}

Additionally, this knowledge can be used to calculate percent composition of subtances from its molecular formulas.

\subsubsection{Percent Composition from Known Formulas}

\begin{example}
For example, given the formula \ce{MgCl2} and the atomic masses of \ce{Mg} and \ce{Cl} at 24.30 amu and 35.45 amu respectively, one can compute the total molecular mass of \ce{MgCl2} as $1 \times 24.30 + 2 \times 35.45 = 95.20 \mathrm{amu}$, and from there, the percent composition of each element as $\ds \frac{24.30}{95.20} \times 100 = 25.53\%$ for \ce{Mg} and $\ds \frac{2 \times 35.45}{95.20} \times 100 = 74.47\%$ for \ce{Cl}.
\end{example}

\subsubsection{Percent Composition from Experimental Data}

Additionally, when calculating percent compositions from unknown compositions with experimental data, simply manually calculate the total mass from the data then divide by the mass of each element.

\subsection{Counting Atoms}

\begin{defn}
Large numbers, such as the number of atoms in a substance, are counted in units known as moles, where \textbf{one mole is equal to $6.022 \times 10^{23}$}, a value known as \textbf{Avogadro's number}. Moles are abbreviated as \textit{mol}.
\end{defn}

\noindent
The mass of each mole of an element (or every $6.022 \times 10^{23}$ atoms of the element combined) is known as the \textbf{molar mass}, and it is different for each element. For all elements, the molar mass is equal to the atomic mass when the element is present in an atomic form. As such, atomic mass is the same as atomic molar mass. \\

\noindent
Molar mass of a compound is measured in grams and is numerically equal to the formula or molecular mass of the compound (which is measured in \textit{amu}). For example, the formula/molecular mass of water is 18.02 amu and the molar mass is 18.02 grams.

\subsection{Empirical Formula vs. Molecular Formula}
\begin{defn}
The empirical formula of a compound gives the simplest whole number ratio of the atoms present for each formula unit in a compound, whereas the molecular formula represents the total number of atoms of each element present in one molecule of a compound (which is \textbf{not simplified}).
\end{defn}

\noindent
\note{The molecular formula of a substance is the true formula of the compound, and it is calculated by using the empirical formula and the molar mass together.}

\begin{example}
The empirical formula for Ethylene is \ce{CH2}, but given its molecular mass, the actual molecular formula is \ce{C2H4}.
\end{example}

\subsubsection{Calculating Empirical Formulas}

\begin{enumerate}
\item Assume some definite starting quantity (like 100.0g) of the compound if it isn't given and then calculate the mass of each element component in grams.
\item Following this conversion formula (where $\ds m_{\text{s}}$ is the empirical mass of the element within the substance, $\ds m_{\text{a}}$ is the atomic molecular mass of the element, and $\ds x$ is the number of moles of atoms of this element within the substance,

\begin{equation}
x = \frac{m_{\text{s}}}{m_{\text{a}}}
\end{equation}

\item Divide the moles of atoms of each element by the moles of atoms of the element that had the smallest value. (If they are whole numbers, use them as subscripts and write the empirical formula).
\item Otherwise, multiply the values obtained previously by the smallest numbers that will convert them to whole numbers.
\end{enumerate}

\begin{example}
\textit{An analysis of salt shows that it contains 56.58\% Potassium (\ce{K}), 8.68\% Carbon (\ce{C}), and 34.73\% Oxygen (\ce{O}). Calculate the empirical formula for this substance.} \\

\noindent
Step 1: Express each element in grams, assuming we have 100 grams of the compound.

\begin{align*}
\ce{K} &= 56.58\mathrm{g} \\
\ce{C} &=  8.68\mathrm{g} \\
\ce{O} &= 34.73\mathrm{g}
\end{align*}

\noindent
Step 2: Convert the grams of each element into moles.

\begin{align*}
56.58 \mathrm{g K} \left( \frac{1 \mathrm{mol K}}{39.10 \mathrm{g K}} \right) = 1.447 \mathrm{mol K} \\
8.68 \mathrm{g C} \left( \frac{1 \mathrm{mol C}}{12.01 \mathrm{g C}} \right) = 0.723 \mathrm{mol C} \\
34.73 \mathrm{g O} \left( \frac{1 \mathrm{mol O}}{16.00 \mathrm{g O}} \right) = 2.171 \mathrm{mol O}
\end{align*}

\noindent
Step 3: Divide each number of moles by the smallest molar value calculated previously (0.723 mol).

\begin{align*}
K = \frac{1.447 \mathrm{mol}}{0.723 \mathrm{mol}} = 2.00 \\
C = \frac{0.723 \mathrm{mol}}{0.723 \mathrm{mol}} = 1.00 \\
O = \frac{2.171 \mathrm{mol}}{0.723 \mathrm{mol}} = 3.00
\end{align*}

\noindent
Therefore, the simplest ratio of K:C:O is 2:1:3, resulting in an empirical formula of \ce{K2CO3}.

\end{example}

\subsubsection{Calculating Molecular Formulas}

The molecular formula can be calculated form the empirical formula if the molar mass is known, given that the mass of the molecular formula will be equal to some multiple of the empirical mass. \\

\noindent
The simplest way to calculate the molecular formula is to divide the molar mass of the substance by the empirical mass to determine the scaling value, then multiply all of the coefficients of the formula by this number.

\begin{equation}
n = \frac{\text{Molar Mass}}{\text{Empirical Mass}}
\end{equation}

\note{Where $n$ is the number of empirical formula units.}

\begin{example}
\textit{What is the molecular formula of a compound which has an empirical formula of \ce{CH2} and a molar mass of 126.2 grams?} \\

\noindent
As we know the empirical mass of \ce{CH2} to be 14.03 grams per formula unit, we can calculate

$$n = \frac{126.2 \mathrm{g}}{14.03 \mathrm{g}} = 9$$

\noindent
therefore, the molecular formula is scaled from \ce{CH2} to \ce{(CH2)9} or \ce{C9H18}.
\end{example}

\section{Lecture 2: Stoichiometry}

\begin{defn}
Chemical Stoichiometry is the study of quantitative and molecular relationships between reactants and products within chemical reactions.
\end{defn}

\subsection{Calculating Theoretical Yield}

\begin{defn}
The \textbf{theoretical yield} is the maximum amount of product you will get if the reaction occurs with complete efficiency.
\end{defn}

\noindent
The theoretical yield of a reaction is calculated using the coefficients of a chemical equation, which not only describe the number of atoms/molecules being reacted/produced but also describe the number of moles of atoms/molecules being reacted/produced.

\begin{example}
\textit{Bismuth (III) chloride will react with hydrogen sulfide to form Bismuth (III) sulfide and Hydrochloric acid. Write the balanced equation for this reaction then calculate how many moles of acid would be formed if 15.0 mol of Hydrogen sulfide react.} \\

\noindent
\ce{2BiCl3 + 3H2S -> Bi2S3 + 6HCl}

$$15.0 \text{ mol \ce{H2S}} \times \frac{6 \text{ mol \ce{HCl}}}{3 \text{ mol \ce{H2S}}} = 30.0 \text{ mol \ce{HCl}}$$
\end{example}

\subsection{Percent Yield}

\begin{defn}
The \textbf{actual yield} is the actual amount of product obtained from a chemical reaction and the \textbf{percent yield} is the ratio of the actual yield to the theoretical yield (expressed as a percentage).

\begin{equation}
\text{Percent Yield} = \frac{\text{Actual Yield}}{\text{Theoretical Yield}} \times 100
\end{equation}
\end{defn}

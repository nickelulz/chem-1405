\section{Lecture 1: Reactions}

\subsection{Introduction to Reactions}

\textbf{References Chapter 7, Sections 7.6 and 7.11 of the textbook.}

\begin{defn}
A chemical reaction is the process in which at least one or more new substances are produced as a result of a chemical change.
\end{defn}

\noindent
Evidence of chemical changes include a change in color, temperature, or odor, or the formation of a gas or insoluble solid (or precipitate).

\begin{defn}
The input/starting substances of a reaction are known as \textit{reactants}, and the output/ending substances of a reaction are known as \textit{products}.
\end{defn}

\subsubsection{The Law of Conservation of Mass}

\begin{defn}
Mass is neither created nor destroyed (in any ordinary chemical reaction). Therefore, the total mass of reactants is always equal to the total mass of products.
\end{defn}

\subsection{Chemical Equations}

\begin{defn}
A chemical equation is a representation for a chemical reaction that uses chemical symbols and chemical formulas to describe changes that occur in a chemical reaction.
\end{defn}

\noindent
In a chemical equation, the reactants are written on the left side (before the arrow), and products are written on the right side (after the arrow), with each substance separated by a plus sign on each side.

\begin{example}
\ce{2C3H7OH + 9O2 -> 6CO2 + 8H2O}
\end{example}

\noindent
In order for a chemical equation to be valid, it must satisfy two conditions:

\begin{enumerate}
\item Only the reactants and products involved in the reaction are represented in the equation.
\item Accurate or correct formulas must be used, not simply empirical formulas. For example, diatomic molecules must always be presented as diatomic rather than as single atoms.
\item The same number of atoms of each kind must be present on each side of the equation.
\end{enumerate}

\noindent
In a chemical formula, the physical state of a substance may be indicated using symbols in parentheses.

\begin{example}
\ce{2C3H7OH_{(s)} + 9O2_{g} -> 6CO2_{g} + 8H2O_{(l)}}
\end{example}

\note{\textit{aq} means aqueous.}

\subsubsection{Balancing Equations}

To balance an equation, adjust the number of atoms of each element so that they are the same on each side of the equation. \\

\noindent
\note{When balancing an equation, only change the coefficients on each side. \textbf{NEVER} alter a substance's formula (i.e. subscripts) to balance the equation, as it creates a completely new substance.} \\

\noindent
\note{Remember that the coefficients in a balance equation are the lowest whole numbers that balance the equation. As such, it's incredibly useful to consider polyatomic ions as single entities, provided that they maintain their identities during the reaction.}

\noindent
When balancing equations, follow the order:

\begin{enumerate}
\item Balance the metals
\item Balance the polyatomic ions if they stay together
\item Balance the non-metals
\item Balance hydrogen
\item Balance oxygen
\end{enumerate}

\noindent
\note{Additionally, it also helps to create a table when balancing an equation.}

\begin{example}
\ce{C4H10_{(g)} + O2_{(g)} -> CO2_{(g)} + H2O_{(l)}}

\begin{table}[H]
\centering
\begin{tabular}{|c|c|c|}
\hline
Elements & Reactant Count & Product Count \\
\hline
Carbon   & 4  & 1 \\
Hydrogen & 10 & 2 \\
Oxygen   & 2  & 3 \\
\hline
\end{tabular}
\end{table}

\noindent
Naturally, this process can be followed to reach a balanced state. \\

\ce{2C4H10_{(g)} + 13O2_{(g)} -> 8CO2_{(g)} + 10H2O_{(l)}}

\begin{table}[H]
\centering
\begin{tabular}{|c|c|c|}
\hline
Elements & Reactant Count & Product Count \\
\hline
Carbon   & 8  & 8  \\
Hydrogen & 20 & 20 \\
Oxygen   & 26 & 26 \\
\hline
\end{tabular}
\end{table}
\end{example}

\subsection{Types of Chemical Reactions}

\begin{table}[H]
\centering
\begin{tabular}{|c|c|}
\hline
Type of Reaction & General Equation \\
\hline
Combination (Synthesis)	& \ce{A + B -> AB} \\
Decomposition & \ce{AB -> A + B} \\
Single Replacement & \ce{A + BC -> AC + B} \\
Double Replacement & \ce{AB + CD -> AD + BC} \\
Combustion & \ce{C_{x} + H_{y} + O2 -> CO2 + H2O} \\
\hline
\end{tabular}
\end{table}

\begin{itemize}
\item Synthesis: Two or more reactants combining to form only one product.
\item Decomposition: One reactant breaks down to produce two or more difference substances.
\item Single Replacement: One element reacts with a compound to replace one of the elements of that compound.
\item Double Replacement: Two element react with a compound to replace two of the elements of that compound.
\item Combustion: Substance reacts with oxygen and produces heat and/or a flame.
\end{itemize}

\noindent
\note{A double replacement reaction will usually result in one or more of these three: the formation of a precipitate, the formation of a gas/bubbles, the release or absorption of heat, or the formation of a molecular compound (such as \ce{H2O}).}
